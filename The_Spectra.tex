\subsection{The Spectra}

\iffalse
	{\bf List of stars with multiple spectral type references in literature with notes:}\\
	\begin{itemize}
		\item{} CD-25 11942
		\item{}~~~match isn't great

		\item{} GSC 06213-00306AB
		\item{}~~~missing exact match in comparison plot

		\item{} GSC 06793-00797
		\item{}~~~match isn't great

		\item{} GSC 06793-01406
		\item{}~~~missing exact match in comparison plot

		\item{} GSC 06801-00186
		\item{}~~~missing exact match in comparison plot

		\item{} HIP 78977
		\item{}~~~could be F7 or F8

		\item{} HIP 79369
		\item{}~~~could be F1 or F0

		\item{} ScoPMS 44
		\item{}~~~match isn't great

		\item{} ScoPMS 214
		\item{}~~~match isn't great
	\end{itemize}
\fi

[See commented-out list above]\\


Of the 46 stars composing this atlas, 9 have more than one spectral type 
referenced in the literature.   For such stars, spectral classes referenced 
in Table~\ref{tab:maintab} were determined by visual examination of individual 
spectra.  Observed spectra were compared to existing libraries, with established 
spectral classes.  References were chosen based on EW values and comparison of 
spectral features.\\


\noindent [{\bf section 3.1 Rayner}]\\
\noindent [Digital copies of the spectra, with associated information, can be found at $----$.]\\
\noindent [Each spectra is available as a fits or png file.]\\
\noindent [Discuss masking of telluric regions.]\\



