Calibration frames, flats and arcs, were taken immediately after collecting data on a particular science object.  The final step was to obtain the remaining calibration data, pertaining to a particular A0.  Which standard to observe was determined by location and airmass.  For our purposes, an ideal A0V would deviate from the science objects' airmass by no more than 0.15 and be located in the same region of the sky as the science object.  This ensured minimal atmospheric derivations between our science and A0 stars.



Comparison of the data taken on each A0V standard to that of a well known A0, Vega, allowed for the determination of atmospheric residuals.  %Spextool made this data reduction possible.  
For reduction purposes, the Spex pipeline was used\footnote{\url{http://irtfweb.ifa.hawaii.edu/~spex/SpeX_manual_06mar15.pdf}}.  The difference between an observed A0 and the model for Vega can be attributed to atmospheric conditions existing at the time of observation.  The same atmospheric disturbances apply to all objects observed at the same time and airmass.  Telluric corrections can now be applied to the spectra of each science object.  During observations, integration times were altered as to maximize the Signal to Noise ratio (SNR). \cite{Cushing_2004}\\