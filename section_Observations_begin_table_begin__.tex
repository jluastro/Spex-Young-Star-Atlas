\section{Observations}

%\begin{table}
%\begin{tabular}{ccc}
%Verion (SXD mode) & Wavelength Range & R (0.3" slit) \\
%Spex (Pre-upgrade) & 0.80-2.4 $\mu$ m & 2000 \\
%Spex (Post-upgrade) & 0.70-2.55 $\mu$ m & 2000 \\
%\end{tabular}
%\end{table}




%Included in this update was an increase in precision, permitting Spex to take a larger number of wavelength samplings...This update increased the rate at which Spex samples wavelengths.  
Spex was updated in August, 2014.  This update increased Spex's wavelength sampling rate, allowing for higher accuracy of collected data.  
%A portion of the objects in this catalog were observed prior to the implementation of this update.  
A portion of the objects in this catalog were observed prior to this update.  For this reason, it is necessary to compare data taken with each version of Spex, shown in Figure \ref{fig:uSpex-Spex}.  
%Figure \ref{fig:uSpex-Spex} shows such a comparison, using GSC 06801-00186.  This object...It is clear from this figure that while the data taken with uSpex is more comprehensive, spectra collected using Spex are still sufficiently accurate.
GSC 06801-00186 was observed on June 29, 2012 UT, before Spex was updated, and again on June 15, 2015, after uSpex was implemented \cite{Spextool_Manual_Cushing_2015}.  From this figure, it is clear that data collected with either version is sufficiently accurate, but uSpex spectra are more comprehensive.  Data collected with both versions follow the same procedure, discussed above.  Before the upgrade, the wavelength range spanned 0.80-2.4 $\mu$m.  Following August 2014, the wavelength range was expanded to span 0.70-2.55~$\mu$m.  Dates of observations for particular targets are displayed in Table \ref{tab:maintab}. \\



[SNR calculated over range of 2.025-2.162 microns]


%%%% FOR TABLE:
%\caption{Listed above are observed targets with corresponding information}\label{tab:maintab}
%\end{tabular}
%\end{table}