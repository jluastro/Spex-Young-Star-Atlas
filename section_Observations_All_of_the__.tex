\section{Observations}



All of the objects discussed in this paper were observed with the NASA Infra-Red Telescope Facility (IRTF) and the SpeX instrument \cite{Rayner_1998} in the short-wavelength cross-dispersed mode (SXD).  During observations, we collected two distinct types of data.  One set before Spex was updated and another after.  The most recent upgrade to Spex occurred in August 2014.  This upgrade increased the observable wavelength.  See table 1 for comparison.

[HOW DO I PROPERLY CITE WAVELENGTH RANGE IN TABEL]\cite{RAYER_SPEX_OBSERVING_MANUAL_or_Spextool_2015_manual}.\\

Fubar

Calibration frames, flats and arcs, were taken immediately after collecting data on a particular science object.  The final step was to obtain the remaining calibration data, pertaining to a particular A0V standard.  Which A0 to observe was determined by location and airmass.  For our purposes, an ideal standard would deviate from the science objects' airmass by no more than 0.15 and be located in the same region of the sky as the science object.  This ensured minimal atmospheric derivations between our science and A0V stars.

\begin{table}
\begin{tabular}{ccc}
Verion (SXD mode) & Wavelength Range & R (0.3" slit) \\
Spex (Pre-upgrade) & 0.80-2.4 $\mu$ m & 2000 \\
Spex (Post-upgrade) & 0.70-2.55 $\mu$ m & 2000 \\
\end{tabular}
\end{table}




Observations were made in AB pairs.  After the initial A frame is taken, the telescope offsets ("nods") and captures a B frame with the same science target at a different position along the slit.  Since our objects were treated as point sources, this AB mode allows for the subtraction of the B frame from the A frame, A-B, leaving both positive and negative spectrum along with sky residuals \cite{Cushing_2004}. Subtraction of these pairs allows for the removal of dark currents and sky residuals \cite{Joyce_1992}.\\


After collecting data on a particular science object, calibration frames were taken.  Before altering the telescope in anyway, flats and arcs were taken.  By not moving the telescope the time between observing an object and taking its corresponding calibration frames could be minimized.  Such calibration frames are used to identify and subtract off noise generated by the detector.\\



Compairison of the data taken on each A0 standard to that of a well known A0, Vega, allowed for the determination of atmospheric residuals.  Using Spextool made data reduction, in this manner, possible.  The difference between an observed A0 and the model for Vega can be attributed to atmospheric conditions existing at the time of observation.  The same atmospheric disturbances apply to all objects observed at the same time and airmass.  Telluric corrections can now be applied to the spectra of each science object.  During observations, integration times were altered as to maximize the Signal to Noise ratio (S/N). \cite{Cushing_2004}\\


In August 2014, Spex was updated.  Included in these updates was an increase in precision, which permitted Spex to take a larger number of  wavelength samplings.  This allowed for higher accuracy of collected data.  A portion of the objects in this catalog were observed prior to the update occurring.  For this reason it is necessary to compare data taken with each version of Spex. Figure 1 shows such a comparison for the object GSC 06801-00186.  This object was observed on June 29, 2012 UT, before Spex was updated, and again on June 15, 2015, after uSpex was implemented \cite{Spextool_Manual_Cushing_2015}.  It is clear from this figure that while the data taken with uSpex is more comprehensive, spectra collected using Spex are still sufficiently accurate.  Data collected with both Spex and uSpex follow the same procedure discussed above.\\

