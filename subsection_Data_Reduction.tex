\subsection{Data Reduction}

[ORDERING: aov for telluric...the choice of ao is standard...the obs a0 was comp vega]
%-----
Calibration frames consist of flats, arcs, and observed A0V data.  Flat frames allow for the removal of inconsistencies, amongst the detector's pixels.  Observations of an arc lamp permitted wavelength calibration.  The choice of A0V stars as standards was based on their relatively few spectral features, outside hydrogen lines; making isolation of telluric lines significantly cleaner.  
For telluric reduction, an observed A0V star was compared to Vega.  Deviations of the observed star, from the standard, are attributed to atmospheric interference.  Telluric corrections were accomplished using spectroscopic observations of standard stars. B-V data, provided by Simbad~\cite{simbad}, was used in in the standard selection process.
In order to properly scale emission lines and account for velocity shifts, a kernel was constructed using the observed A0V.  Finally, all orders were scaled and merged, producing a continuous spectrum.
A more detailed account of this process is outlined by Vacca~\cite{Vacca_2003}.  
This process allows for such telluric residuals to be removed.  For reduction of collected spectra, Spextool was used~\cite{Cushing_2004}.  


To be transformed from an array into a workable spectrum Spextool~\cite{Cushing_2004} was used.  Once extracted, each spectra was visually reviewed.  Hot pixels, outliers, and areas of low SNR were masked and removed.  Through this process, the intrinsic spectrum of each star was better revealed.  SNR calculations occurred between 2.025-2.162~$\mu$m.  All stars in this sample have SNR between 100 -- 210.