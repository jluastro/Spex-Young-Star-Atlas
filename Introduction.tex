\section{Introduction}
  



  
%We want to write a paper like John Rayner's \cite{Rayner_2003}
%  \cite{Vacca_2003}
  
Upper Scorpius (Upper Sco) is a star forming region located in 
the Scorpius–-Centaurus Association.  Members of star forming regions 
are born at approximately the same time.  Upper Sco has an established age 
of $\sim$11~Myr~\cite{Pecaut_2012}.  Old stars are currently used in 
identifying spectral type~\cite{Rayner_2009, Ivanov_2004}.  With ages on 
the order of billions of years, these stars do not accurately represent 
young stars.
%The SpeX Atlas currently uses old stars to classify spectral types. Many of these stars are billions of years old~\cite{Rayner_2009}.  

%Mark J. Pecaut 2012: A REVISED AGE FOR UPPER SCORPIUS AND THE STAR FORMATION HISTORY AMONG THE F-TYPE MEMBERS OF THE SCORPIUS–CENTAURUS OB ASSOCIATION


Observations made of young stars exhibit unexpected 
spectral features.  These variations prove young stars 
stars need independent spectral classification.  A stellar 
atlas of young stars does not exist at infrared (IR) 
wavelengths.  IR observations cut through gas and dust found 
in star forming regions.



Presented here is an atlas of young stars, allowing for 
more accurate classification.  Criteria required in building such 
an atlas is discussed in Section~\ref{sec:sampsel}.  
This atlas will allow for the refinement of stellar evolution and 
atmosphere models.  Doing so will allow for more accurate spectral 
classification of young stars.


%\iffalse
%[Need to add introduction to Upper Sco here.]\\
%~[start at same level as UROP - difference between old and young stars]\\
%~[why do we need young atlas]\\
%\fi
[young star - surface gravity?]\\
~[why ir wavelength]\\
~[future prospects]