\section{authorea comments}

**old observation section:

#>>May want to reorganize sections 2, 3, and 4 a bit. Put data reduction stuff under observations; reserve the “Analysis” section for the actual analysis you do with the final reduced spectra

%Spectra collected after the update are comprehensive,
%#>>What do you mean by this? Comprehensive in that all the stars in our sample are observed? Or comprehensive in that the wavelengths cover everything we want?


% Standard A0V stars also needed to be observed.
%#>>State purpose: for telluric line corrections

 %Dates of observations for particular targets are displayed in Table 1.
 %#>>Maybe do something like: "Observations were obtained between XXX and XXX (Table 1)"



 %**old Data Reduction and Analysis:

%  The most recent upgrade to Spex occurred in August 2014. This upgrade increased the observable wavelength.
%  #>> Be sure to mention wavelength gap in old Spex (about 1.8 microns), but not in uSpex

%   All of the objects discussed in this paper were observed with the NASA Infra-Red Telescope Facility (IRTF) and the SpeX instrument (Rayner 1998) in the short-wavelength cross-dispersed mode (SXD), with a resolution of 2000. During observations, we collected two distinct types of data. One set before Spex was updated and another after. The most recent upgrade to Spex occurred in August 2014. This upgrade increased the observable wavelength. See Table ??? for comparison.
%   #>> Repeated information...m...

% a kernel was constructed using the observed A0
%   #>> What does this mean? Could use more details here

%   During observations, integration times were altered as to maximize the Signal to Noise ratio (SNR)
%   #>> This belongs much earlier, near the beginning of the observations section. This section should read like a story...once the observations are taken and the calibrations are done, we can't go back and alter integration times, right? We do that at the telescope. So, this detail must be earlier in the section



%%
% 6/22/16
%%

%% Section 2.1 - Sample Selection
To better classify young stars (∼11 Myr old), the most essential restriction on this atlas was to have each star be approximately the same age.
#>> Since you will have introduced Upper Sco in Section 1, you can just get straight to the sample selection details. Delete the first sentence and replace with something like: "We selected NNN members of the Upper Scorpius star forming region spanning spectral types from OOO to MMM...

properly
#>> Usually you want to avoid words like "properly" since this is subjective. Instead say "was vetted using the following criteria." or some such thing.

%immediately
%#>> remove

s possible
#>> Add in more details as to where the spectral types came from.

 catalogs in the literature
 #>> you can't just say this. You have to insert the references. If you want to keep them only in the table, then the reference to the table needs to be in this sentence.


%% Histogram
%This should be in temperature order, not letter order. BAFGKM, with S on the end as an oddity

%% Section 2.2: Observations

% ),
% #>> no comma

% t1
% #>> Put the footnote right by SpeX

Observations were made using the Spex Observing Manual 2
#>> This sentence doesn't make sense. You can't make observations with a manual. Rephrase.

upgrade
#>> what upgrade?

%August 2014
%#>> Is there a reference for this upgrade?

A portion of the objects in this catalog were observed prior to this upgrade. For this reason, it is necessary to compare data taken with each version of Spex (Table 1), shown in Figure 2.
#>> This should start a new paragraph and go with the sentences on GSC 06801-00186 where the pre and post upgrade data are being compared. I suggest moving the whole pre/post upgrade comparison below the data analysis paragraphs.

% we used
%#>> present or past tense?


%% Table
We might want to think about re-ordering this table so that the stars are in temperature order, rather than HIP number. Note that Table 2 of Rayner+09 is in temperature order

%% Plot comparing Spex, uSpex
Probably doesn’t need to be a 2 panel plot...what does the top panel show that the bottom doesn’t? Consider just the bottom panel, with a legend showing red = spex and green = uspex

%% Stacked plots
Order of spectra is different in this plot (hottest - coldest) compared to other plots (coldest - hottest). Rayner+09 goes from hottest - coldest, so should probably do it that way