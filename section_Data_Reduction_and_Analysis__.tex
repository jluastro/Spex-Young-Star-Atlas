%\section{Data Reduction and Analysis}

All of the objects discussed in this paper were observed with the NASA Infra-Red Telescope Facility (IRTF) and the SpeX instrument \cite{Rayner_1998} in the short-wavelength cross-dispersed mode (SXD), with a resolution of 2000.  During observations, we collected two distinct types of data.  One set before Spex was updated and another after.  The most recent upgrade to Spex occurred in August 2014.  This upgrade increased the observable wavelength.  See table 1 for comparison.

%[HOW DO I PROPERLY CITE WAVELENGTH RANGE IN TABEL]\cite{RAYER_SPEX_OBSERVING_MANUAL_or_Spextool_2015_manual}.\\

Calibration frames, flats and arcs, were taken immediately after collecting data on a particular science object.  The final step was to obtain the remaining calibration data, pertaining to a particular A0V standard.  Which A0 to observe was determined by location and airmass.  For our purposes, an ideal standard would deviate from the science objects' airmass by no more than 0.15 and be located in the same region of the sky as the science object.  This ensured minimal atmospheric derivations between our science and A0V stars.
