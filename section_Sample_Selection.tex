\section{Sample Selection}

This database was built to better classify young stars (less than ~10Myr old), making it essential to restrict this atlas to stars of approximately the same age.  
%One of the top places to look for stars that formed at approximately the same time is in star forming region, solidifying the choice of observing upperSco stars.
%One of the top places to look for such stars is in star forming region, solidifying the choice of observing upperSco stars.
Star forming regions are ideal locations for such stars, solidifying the choice of observing upperSco members.  Various parameters had to be met by each star before it was chosen for observation.  In order to build a comprehensive library, a wide spectral class range needed to be spanned (from class M to class O).  




%Stars classified as having binary companions were immediately eliminated \cite{binary_guy}.  Finally, any star that had been determined to have an accretion disk \cite{disk_guy} was also removed from the target list.  Restricting the selection of target objects based on such criteria ensures each observed spectrum is as isolated and representative as possible. 
Spectral classes were determined after careful analysis of collected spectra.  Stars identified to have binary companions \cite{binary_guy} or accretion disks \cite{binary_guy} were immediately eliminated from the target list.
%Finally, any star that had been determined to have an accretion disk \cite{disk_guy} was also removed from the target list.
Restricting target objects based on such criteria ensures each observed spectra was as isolated and representative as possible.


Observations were made in AB pairs.  After the initial A frame is taken, the telescope offsets ("nods") and captures a B frame.  Both frames are of the the same science target, at a different position along the slit.  Since our objects were treated as point sources, this AB mode allowed for the subtraction of the B frame from the A frame, leaving both positive and negative spectrum along with sky residuals \cite{Cushing_2004}. Subtraction of these pairs allows for the removal of dark currents and sky residuals.



After collecting data on a particular science object, flat and arc calibration frames were taken.  
%Before altering the telescope in anyway, flats and arcs were taken.  By not moving the telescope, the time between observing an object and taking its corresponding calibration frames was minimized.  Such calibration frames were used to identify and subtract off noise generated by the detector.

In order to minimize the time between target observations and the collection of calibration frames, the telescope remained unmoved.  Noise, generated by the detector, was identified and removed using calibration frames.