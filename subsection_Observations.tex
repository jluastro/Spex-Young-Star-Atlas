\subsection{Observations}

All of the objects discussed in this paper were observed between 
March 2012 and June 2015 (Table~\ref{tab:maintab}), with the 
NASA Infra-Red Telescope Facility (IRTF) and the SpeX 
instrument~\cite{Rayner_1998}. We used the short-wavelength cross-dispersed 
mode (SXD) with $R ∼ 2000$ matched to $0.3x15"$ slit.  Spex was upgraded in August 2014\footnote{See \url{http://irtfweb.ifa.hawaii.edu/~spex/SpeX_manual_06mar15.pdf} for details.}. 
%For reduction purposes, the Spex pipeline was used\footnote{\url{http://irtfweb.ifa.hawaii.edu/~spex/SpeX_manual_06mar15.pdf}}.  
%This upgrade increased Spex's wavelength sampling rate, allowing for higher accuracy of collected data 
This upgrade increased the observable wavelength range, filled 
in the gap around 1.8$\mu$m, and increased Spex's wavelength 
sampling rate, allowing for higher accuracy of collected data.
A portion of the objects in this catalog were observed prior to this upgrade.  
For this reason, it is necessary to compare data taken with each version of Spex (Table~\ref{tab:maintab}), 
shown in Figure~\ref{fig:uSpex-Spex}.
%, as shown in Figure~\ref{fig:uSpex-Spex}.  See Table~\ref{tab:maintab} for comparison.



GSC 06801-00186 was observed on June 29, 
2012 UT, before Spex was upgraded, and again on June 15, 2015, after uSpex 
was implemented~\cite{Spextool_Manual_Cushing_2015}.  Spectra collected after 
the upgrade span a larger wavelength range, but both Spex and uSpex data sufficiently 
cover the wavelength range needed for our study.  Data collected with both versions 
follow the same procedure, discussed below.  Before the upgrade, the wavelength 
range spanned 0.80-2.4~$\mu$m.  Following August 2014, the wavelength range 
was expanded to span 0.70-2.55~$\mu$m.\\

%  During observations, we collected two distinct types of data.  One set before Spex was updated and another after.  
 

%[HOW DO I PROPERLY CITE WAVELENGTH RANGE IN TABEL]\cite{RAYER_SPEX_OBSERVING_MANUAL_or_Spextool_2015_manual}.\\


During observations, integration times were altered as to maximize the Signal to Noise ratio (SNR).  
Observations were made in AB pairs.  After the initial A frame is taken, the telescope offsets ("nods") and captures a B frame.  Both frames are of the the same science target, at a different position along the slit.  Since our objects were treated as point sources, this AB mode allowed for the subtraction of the B frame from the A frame, leaving both positive and negative spectrum along with sky residuals~\cite{Cushing_2004}. Subtraction of these pairs allows for the removal of dark currents and sky residuals.



After collecting data on a particular science object, 
flat and arc calibration frames were taken.  
In order to minimize the time between target observations and 
the collection of calibration frames, the telescope remained unmoved.  
Background noise was identified and removed using darks and flats.



Standard A0V stars also needed to be observed, for telluric line corrections.  Which A0V 
to observe was determined by location and airmass.  For our purposes, an ideal A0V would 
deviate from the science objects' airmass by no more than 0.15 and be located 
in the same region of the sky as the science object.  This ensured minimal 
atmospheric derivations between our science and A0V stars.
%The airmass difference between a target and corresponding A0 never exceed 0.15.  
%Airmass of each A0 star differed from that of the targets by no more than 0.15.

