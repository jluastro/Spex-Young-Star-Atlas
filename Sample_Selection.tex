\subsection{Sample Selection}\label{sec:sampsel}

%\iffalse
%%Any stellar atlas needs to be comprehensive.
%To better classify young stars ($\sim$11~Myr old), 
%the most essential restriction on this atlas was to 
%have each star be approximately the same age. 
%%This database was built to better classify young stars ($\sim$11~Myr old), making it essential to restrict this atlas to stars of approximately the same age. 
%%To better classify young stars ($\sim$11~Myr old), the more essential restriction on this atlas was that all the stars were approximately the same age.
%Star forming regions are ideal locations for such stars, 
%solidifying the choice of observing Upper Sco members.  
%Surveying the literature verified all targets as members of this 
%region.  In order to be build a comprehensive atlas of young 
%stars, various spectral types needed to be included 
%(from class M to class O).
%%~[HOW DO WE KNOW THESE ARE UPPERSCO MEMBERS (DO A LIT SEARCH)]\\
%\fi

%51 members total
% 47 if you include speX and uspeX
We selected 46 members of the Upper Scorpius star forming region spanning spectral types from M--O.
%~[NEED TO ASK JESSICA, HOW THEY DETERMINED WHAT STARS WERE BINARY & WHICH HAD DISKS]\\
Prior to observation, each target was vetted using the following criteria. 
Stars identified to have binary companions~\cite{binary_guy} or accretion 
disks~\cite{binary_guy}, were eliminated from the target list.  
Restricting target objects based on such criteria ensures each observed 
spectra was as isolated and representative as possible.



%[HOW WERE PREVIOUS CLASSES DETERMINED (CONFIRM OPTICAL)]\\
To select potential targets, previously established spectral classes 
were used.  
%Sifting through catalogs in the literature uncovered established spectral classes, at optical wavelengths.
Observations at optical wavelengths established 
spectral types, listed in Table~\ref{tab:maintab}.  
%References for each previous spectral classification can be found in Table~\ref{tab:maintab}.\\




%In ensure chosen targets span the necessary spectral class range, 
%previously established spectral classes were used.  The referenced 
%spectral classes were determined using optical observations.
%%To select potential targets, previously established spectral classes were used.  
%References for each stars optical classification 
%can be found in Table~\ref{tab:maintab}.\\
%~[INCLUDE HISTOGRAM OF N_STARS VS SPT TYPE]







