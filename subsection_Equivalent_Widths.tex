\subsection{Equivalent Widths}


Spectral features were identified by~\cite{Rayner_2009}.  

\begin{table}[H]
	\begin{tabular}{c|c|c|c}
	\label{tab:features}
		Feature & Limits ($\mu$m) & First Continuum Level Limits ($\mu$m) & Second Continuum Level Limits ($\mu$m) \\ \hline
		Ca II (0.866~$\mu$m) & 0.860--0.875 & 0.862--0.864 & 0.870--0.873 \\
	\end{tabular}
\end{table}

\begin{equation}\label{eq:EW}
	EW = \sum_{i=1}^{n} [1 - \frac{f(\lambda_{i})}{f_{c}(\lambda_{i})}] \Delta\lambda_{i}
\end{equation}
Calculation of equivalent width values was done using Equation~\ref{eq:EW}.  A more in-depth discussion of this procedure is given by~\cite{Cushing_2005}.  \cite{Sembach_1992}~discusses the technique used in estimating $f_{c}(\lambda_{i})$, $\sigma(\lambda_{i})$, and $\sigma_{c}(\lambda_{i})$.

Procedure:\\
\begin{itemize}
	\item{} Define spectral window using
	\item{} Estimate continuum
	\item{}~~~~~Unweighted linear fit to 1st and 2nd Limits from Table~\ref{tab:features}
	\item{} Sum using Eq.~\ref{eq:EW}
\end{itemize}

