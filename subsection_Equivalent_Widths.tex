\subsection{Equivalent Widths}


Spectral features were identified by~\cite{Rayner_2009}.  

\begin{table}[H]
	\begin{tabular}{c|c|c|c}
	\label{tab:features}
		Feature & Limits ($\mu$m) & First Continuum Level Limits ($\mu$m) & Second Continuum Level Limits ($\mu$m) \\ \hline
		Ca II (0.866~$\mu$m) & 0.860--0.875 & 0.862--0.864 & 0.870--0.873 \\
	\end{tabular}
\end{table}

\begin{equation}\label{eq:EW}
	EW = \sum_{i=1}^{n} [1 - \frac{f(\lambda_{i})}{f_{c}(\lambda_{i})}] \Delta\lambda_{i}
\end{equation}

\begin{equation}\label{eq:EWvar}
	\sigma_{EW}^{2} = \sum_{i=1}^{n} \Delta\lambda_{i}^{2} [\frac{\sigma^{2}(\lambda_{i})}{f_{c}^{2}(\lambda_{i})} + \frac{f^{2}(\lambda_{i})}{f_{c}^{4}(\lambda_{i})}\sigma_{c}^{2}(\lambda_{i})
\end{equation}


Calculation of equivalent width values was done using Equations~\ref{eq:EW}~and~\ref{eq:EWvar}~\cite{Cushing_2005}.  A more in-depth discussion of this procedure is given by~\cite{Cushing_2005}.  \cite{Sembach_1992}~discusses the technique used in estimating $f_{c}(\lambda_{i})$, $\sigma(\lambda_{i})$, and $\sigma_{c}(\lambda_{i})$.

Procedure:\\
\begin{itemize}
	\item{} Define spectral window using
	\item{} Estimate continuum
	\item{}~~~~~Unweighted linear fit to 1st and 2nd Limits from Table~\ref{tab:features}
	\item{} Sum using Eq.~\ref{eq:EW}
	\item{}~~~~~Note: sum from 1--N not 0--N
\end{itemize}



Equivalent width (EW) and uncertainty values were calculated using Eq...where $f(\lambda_{i})$ is
$f_{c}(\lambda_{i})$ is
$\sigma(\lambda_{i})$ is
and $\sigma_{c}(\lambda_{i})$ is
Following the procedure outlined/discussed in~\cite{Cushing_2005}, the continuum limits were used in calculating a linear-fit.  For each $\lambda_{i}$ value\\
~[what is the term used in paper matt sent]\\
$f_{c}(\lambda_{i})$ was determined using the linear fit parameters.


$\Delta\lamda$ is [half?] the difference between subsequent wavelength bins.  To ensure wavelength and flux arrays were the same length, $\lambdi_{N}$
~[accounts for sum going from 1 to N, rather than 0 to N]\\
~[~[VERIFY THIS WORKS...STITCHING THE LAST WAVELENGTH VAL ON THE END CAuSES A DIFFERENCE OF 0~]~] ~~(lamb[-1] - lamb[-1]= 0)\\



Refining EW procedure involved \\
Validation of EW values was accomplished by first verifying published values~\cite{Rayner_2009}.  

~\\~[discuss which lines were chosen and why]\\
~[how do they map to spt type...lum class]\\
~[what features are for young stars??]\\
~[comment on the paper matt dug up??...EW verification paper]


