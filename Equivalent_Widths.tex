\subsection{Equivalent Widths}

\indent {\bf I)} Spectral features act as indicators of many stellar properties.  \\
\indent {\bf II)} Stellar properties can be determined through analysis of spectral features.  \\
The spectral features listed in Table~\ref{tab:features} 
%were identified by~\cite{Rayner_2009}
can be used to determine spectral types of cool 
stars~\cite{Rayner_2009}.  Equivalent width (EW) 
values of these features are given in Table~\ref{tab:EWvals}.  
%Equations~\ref{eq:EW}~and~\ref{eq:EWvar} were used in calculating EW values and uncertainties, following the procedure described by~\cite{Cushing_2005}. 
Following the procedure described by~\cite{Cushing_2005}, EW values, $EW$, and 
variances, $\sigma_{EW}^{2}$, are given by
%Calculation of equivalent width values was done using Equations~\ref{eq:EW}~and~\ref{eq:EWvar}~\cite{Cushing_2005}.
%A more in-depth discussion of this procedure is given by~\cite{Cushing_2005}.  \cite{Sembach_1992}~discusses the technique used in estimating $f_{c}(\lambda_{i})$, $\sigma(\lambda_{i})$, and $\sigma_{c}(\lambda_{i})$.


\begin{equation}\label{eq:EW}
	EW = \sum_{i=1}^{n} \bigg[1 - \frac{f(\lambda_{i})}{f_{c}(\lambda_{i})} \bigg] \Delta\lambda_{i},
\end{equation}

\begin{equation}\label{eq:EWvar}
	\sigma_{EW}^{2} = \sum_{i=1}^{n} \Delta\lambda_{i}^{2} \bigg[ \frac{\sigma^{2}(\lambda_{i})}{f_{c}^{2}(\lambda_{i})} + \frac{f^{2}(\lambda_{i})}{f_{c}^{4}(\lambda_{i})}\sigma_{c}^{2}(\lambda_{i}) \bigg],
\end{equation}


\noindent {\bf A)} where $f(\lambda_{i})$ and $f_{c}(\lambda_{i})$ are the observed 
and calculated continuum flux densities, respectively.  Uncertainties in the 
observed and calculated continuum flux densities are $\sigma(\lambda_{i})$ and 
$\sigma_{c}(\lambda_{i})$, respectively, were calculated following the procedure 
described by~\cite{Sembach_1992}.  To estimate these values, \cite{Sembach_1992} 
transformed from wavelength to velocity space.  Rather than subtracting adjacent intervals, 
\cite{Sembach_1992} defined $d\nu = \nu_{i+\frac{1}{2}} - \nu_{i-\frac{1}{2}}$.  
$\Delta\lambda$ is the difference between subsequent wavelength bins; 
$\Delta\lambda = \lambda_{i+1} - \lambda_{i}$.  To subtract adjacent 
wavelength intervals in this fashion, $\lambda_{n}$ was appended to the 
end of the wavelength array, preserving dimensionality of the arrays.  
This slight variation in methodology was shown to produce the same results.



\noindent {\bf B)} where $f(\lambda_{i})$ and $f_{c}(\lambda_{i})$ are the observed 
and estimated continuum flux densities, respectively.  Uncertainties in the 
observed and estimated continuum flux densities, $\sigma(\lambda_{i})$ and 
$\sigma_{c}(\lambda_{i})$, were calculated following the procedure 
of~\cite{Sembach_1992}.  


	\indent {\bf 1)} $\Delta\lambda$ is the difference between subsequent 
	wavelength intervals; $\Delta\lambda = \lambda_{i+1} - \lambda_{i}$.   
	Dimensionality was preserved by appending $\lambda_{n}$ to the end 
	of each wavelength array\\

	\indent {\bf 2)} $\Delta\lambda$ is the difference between subsequent 
	wavelength intervals; $\Delta\lambda = \lambda_{i+1} - \lambda_{i}$.   
	To subtract adjacent wavelength intervals, array dimensionality needed 
	preservation.  This was achieved by appending $\lambda_{n}$ to the end 
	of each wavelength array.\\

	\indent {\bf 3)} $\Delta\lambda$ is the difference between subsequent 
	wavelength intervals; $\Delta\lambda = \lambda_{i+1} - \lambda_{i}$.   
	To preserve array dimensionality $\lambda_{n+1}$ was set to $\lambda_{n}$.\\

	\indent {\bf 4)} With $\Delta\lambda = \lambda_{i+1} - \lambda_{i}$, wavelength 
	arrays weren't properly shaped.  Preservation of dimensionality was achieved 
	by appending $\lambda_{n}$ to the end of each wavelength array.\\


Rather than subtracting adjacent intervals, 
\cite{Sembach_1992} converted from wavelength to velocity--space and defined 
$d\nu = \nu_{i+\frac{1}{2}} - \nu_{i-\frac{1}{2}}$.  This slight variation in 
methodology was shown to produce the same results, within error.



Validation of this procedure was accomplished by calculating EW values and 
uncertainties of existing spectral libraries\footnote{Table 9 of \cite{Rayner_2009}.}.  



\noindent [CAN THIS BE EXPRESSED AS $\lambda_{i+1} - \lambda_{i} == \lambda_{i} - \lambda_{i-1}$]\\
~[THIS ACCOUNTS FOR SUM GOING FROM 0 TO n]



%Following the procedure outlined/discussed in~\cite{Cushing_2005}, the continuum limits were used in calculating a linear-fit.  For each $\lambda_{i}$ value\\
%~[what is the term used in paper matt sent]\\
%$f_{c}(\lambda_{i})$ was determined using the linear fit parameters.


{\bf Procedure:}\\
\begin{itemize}
	\item{} Define spectral window
	\item{} Estimate continuum
	\item{}~~~~~Unweighted linear fit to 1st and 2nd Limits from Table~\ref{tab:features}
	\item{} Sum using Eq.~\ref{eq:EW}
	\item{}~~~~~Note: sum from 1--N not 0--N
\end{itemize}


{\bf Notes:}\\
\begin{itemize}
	\item{} Refining EW procedure involved
	\item{} Validation of EW values was accomplished by first verifying published values~\cite{Rayner_2009}.
	\item{} Discuss difference between Sembach and my method: $\Delta\lambda = \lambda_{i+1} - \lambda_{i}$
	\item{} accounts for sum going from 1 to n, rather than 0 to n
	\item{} VERIFY THIS WORKS...STITCHING THE LAST WAVELENGTH VAL ON THE END CAUSES A DIFFERENCE OF 0
	\item{}~~~~~(lamb[-1] - lamb[-1]= 0)
	\item{} recalc using SS92 $\lambda_{i+\frac{1}{2}}$--$\lambda_{i-\frac{1}{2}}$
	\item{}~~~~~do by using the full window to restrict values, then go from 1 to N in for loop
\end{itemize}




{\bf Discuss:}\\
\begin{itemize}
	\item{} discuss which lines were chosen and why
	\item{} how do they map to spt type...lum class
	\item{} what features are for young stars??
\end{itemize}


