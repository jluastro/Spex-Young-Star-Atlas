\subsection{Equivalent Widths}

Spectral features act as indicators of many stellar properties.  The 
spectral features listed in Table~\ref{tab:features} 
%were identified by~\cite{Rayner_2009}
are used in determining the spectral types of cool 
stars~\cite{Rayner_2009}.  Equivalent width (EW) 
values are given in Table~\ref{tab:EWvals}.  
Equations~\ref{eq:EW}~and~\ref{eq:EWvar} were used in calculating EW values 
and uncertainties, following the procedure described by~\cite{Cushing_2005}. 
%Calculation of equivalent width values was done using Equations~\ref{eq:EW}~and~\ref{eq:EWvar}~\cite{Cushing_2005}.
%A more in-depth discussion of this procedure is given by~\cite{Cushing_2005}.  \cite{Sembach_1992}~discusses the technique used in estimating $f_{c}(\lambda_{i})$, $\sigma(\lambda_{i})$, and $\sigma_{c}(\lambda_{i})$.


\begin{equation}\label{eq:EW}
	EW = \sum_{i=1}^{n} [1 - \frac{f(\lambda_{i})}{f_{c}(\lambda_{i})}] \Delta\lambda_{i}
\end{equation}

\begin{equation}\label{eq:EWvar}
	\sigma_{EW}^{2} = \sum_{i=1}^{n} \Delta\lambda_{i}^{2} [\frac{\sigma^{2}(\lambda_{i})}{f_{c}^{2}(\lambda_{i})} + \frac{f^{2}(\lambda_{i})}{f_{c}^{4}(\lambda_{i})}\sigma_{c}^{2}(\lambda_{i})],
\end{equation}
where $f(\lambda_{i})$ and $f_{c}(\lambda_{i})$ are the observed 
and observed continuum fluxes, respectively.  Uncertainties in the 
observed and continuum fluxes are $\sigma(\lambda_{i})$ and 
$\sigma_{c}(\lambda_{i})$, respectively.  $\Delta\lambda$ is the 
difference between adjacent wavelength intervals~\cite{Sembach_1992}.
%$\Delta\lambda$ is the difference between subsequent wavelength bins; $\Delta\lambda = \lambda_{i+1} - \lambda_{i}$.
%To subtract adjacent wavelength intervals in this fashion, $\lambda_{n}$ was appended to the end of the wavelength array, preserving dimensionality of the arrays.\\


%Following the procedure outlined/discussed in~\cite{Cushing_2005}, the continuum limits were used in calculating a linear-fit.  For each $\lambda_{i}$ value\\
%~[what is the term used in paper matt sent]\\
%$f_{c}(\lambda_{i})$ was determined using the linear fit parameters.


{\bf Procedure:}\\
\begin{itemize}
	\item{} Define spectral window using
	\item{} Estimate continuum
	\item{}~~~~~Unweighted linear fit to 1st and 2nd Limits from Table~\ref{tab:features}
	\item{} Sum using Eq.~\ref{eq:EW}
	\item{}~~~~~Note: sum from 1--N not 0--N
\end{itemize}


{\bf Notes:}\\
\begin{itemize}
	\item{} Refining EW procedure involved
	\item{} Validation of EW values was accomplished by first verifying published values~\cite{Rayner_2009}.
	\item{} Discuss difference between Sembach and my method: $\Delta\lambda = \lambda_{i+1} - \lambda_{i}$
	\item{} accounts for sum going from 1 to n, rather than 0 to n
	\item{} VERIFY THIS WORKS...STITCHING THE LAST WAVELENGTH VAL ON THE END CAUSES A DIFFERENCE OF 0
	\item{}~~~~~(lamb[-1] - lamb[-1]= 0)
	\item{} recalc using SS92 $\lambda_{i+\frac{1}{2}}$--$\lambda_{i-\frac{1}{2}}$
	\item{}~~~~~do by using the full window to restrict values, then go from 1 to N in for loop
\end{itemize}




{\bf Discuss:}\\
\begin{itemize}
	\item{} discuss which lines were chosen and why
	\item{} how do they map to spt type...lum class
	\item{} what features are for young stars??
\end{itemize}


